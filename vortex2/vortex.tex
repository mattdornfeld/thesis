\documentclass[superscriptaddress, onecolumn, prl]{revtex4}

\usepackage{amsmath, mathrsfs}    % need for subequations
\usepackage[pdftex]{graphicx}   % need for figures
\usepackage{verbatim}   % useful for program listings
\usepackage{color}      % use if color is used in text
\usepackage{subfigure}  % use for side-by-side figures
\usepackage{hyperref}   % use for hypertext links, including those to external documents and URLs
\allowdisplaybreaks

\begin{document}

\title{The Unsteady Force Due to Vorticity Creation}
\maketitle
The force that must be applied to a fluid of volume $V$ to generate vorticity $\boldsymbol{\omega}(t)$ inside the volume is 
\begin{equation}
\textbf{F}_{\omega}(t) = \frac{\rho}{2} \frac{\partial}{\partial t} \int_V \textbf{r} \times \boldsymbol{\omega}(t) dV.
\end{equation}
As a simple model of vortex creation we consider the axisymmetric jet emerging from the vocal folds of radius $r_f$ surrounded by a vortical boundary layer of radius $\delta$. As a model for this jet profile we consider the piecewise linear profile shown in Eq. \ref{eq:jet_profile}. This profile possesses three regions. Inside the boundary layer the velocity is constant and equal to $U_0$. The vortical boundary layer has radius $\delta$. Here the velocity rapidly drops from $U_0$ to zero. Outside the boundary the velocity is zero.
\begin{equation}
\label{eq:jet_profile}
\textbf{U} = 
\begin{cases}
U_0 \hat{x} & 0 \leq r < r_f \\
U_0\frac{r_f + \delta - r}{\delta}\hat{x} & r_f \leq r < r_f + \delta \\
0 \hat{x} & r_f + \delta \leq r < r_0
\end{cases}
\end{equation}
From this assumption it can be seen that the mean flow vorticity is $\boldsymbol{\Omega} = \frac{U_0}{\delta} \hat{\phi}$ inside the jet boundary layer and $0$ outside of it. To calculate the unsteady portion of the vorticity let us separate the total flow into the mean flow and a small disturbance $\textbf{u} = \textbf{U} + \epsilon \textbf{u}'$.
\begin{equation}
\textbf{u}' = u_x'(x, r, \phi, t) \hat{x} + u_r'(x, r, \phi, t)  \hat{r} + u_\phi'(x, r, \phi, t)  \hat{\phi}.
\end{equation}
Substituting this assumption into the Euler and continuity equations and gathering terms of order $\epsilon$ we get the linearized Euler and continuity equation for the disturbances
\begin{equation}
\begin{split}
\label{eq:linear_disturbance}
\frac{\partial \textbf{u}' }{\partial t} + U \frac{\partial \textbf{u}'}{\partial x} + u_r' \frac{\partial \textbf{U}}{\partial r} &= -\frac{\nabla p'}{\rho} \\
\nabla \cdot \textbf{u}' &= 0 
\end{split}
\end{equation}
Let's define the Fourier transforms
\begin{equation}
\begin{array}{rrr}
\mathscr{F}_x[f(r,x,\theta,t)] = \int_{0}^{\infty}dx f(r,x,\theta,t) e^{-i \alpha x} \\
\mathscr{F}_{\theta}[f(r,x,\theta,t)] = \int_{0}^{2 \pi}d\theta f(r,x,\theta,t) e^{-i n \theta} \\
\mathscr{F}_{t}[f(r,x,\theta,t)] = \int_{-\infty}^{\infty}dt f(r,x,\theta,t) e^{-i \omega t}
\end{array}
\end{equation}
Let's define the multi dimensional Fourier transform
\begin{equation}
\hat{f}(r, \omega, \alpha, n) = \mathscr{F}[f(r,x,\theta,t)] = \int_{-\infty}^{\infty}dt\int_{0}^{2 \pi}d\theta\int_{0}^{\infty}dxf(r,x,\theta,t) e^{-i (n \theta + \alpha x + \omega t)}
\end{equation}
\begin{equation}
\begin{array}{cccc}
\mathscr{F}[\textbf{u}'] = 
\begin{pmatrix}
\hat{u}_x(r) \\
i \hat{u}_r(r) \\
\hat{u}_\theta(r)
\end{pmatrix}
&
\mathscr{F}[\frac{\partial \textbf{u}'}{\partial t}] = 
\begin{pmatrix}
i \omega \hat{u}_x(r) \\
- \omega \hat{u}_r(r) \\
i \omega \hat{u}_\theta(r)
\end{pmatrix}
&
\mathscr{F}[\frac{\partial \textbf{u}'}{\partial x}] = 
\begin{pmatrix}
i \alpha \hat{u}_x(r) - \mathscr{F}_t [\mathscr{F}_{\theta}[u_0(t)]]  \\
- \alpha \hat{u}_r(r) \\
i \alpha \hat{u}_\theta(r)
\end{pmatrix}
&
\mathscr{F}[\nabla p'] = 
\begin{pmatrix}
i \alpha \hat{p}(r) \\
\hat{p}'(r) \\ 
\frac{in}{r} \hat{p}(r) 
\end{pmatrix}
\end{array}
\mathscr{F}[\nabla \cdot \textbf{u}']
\end{equation}


\begin{equation}
\omega
\begin{pmatrix} 
\hat{u}_x \\
- \hat{u}_r \\
\end{pmatrix}
+ U \alpha  \begin{pmatrix}
\hat{u}_x \\
- \hat{u}_r \\
\end{pmatrix}
+\hat{u}_r \begin{pmatrix}
\frac{\partial U}{\partial r}\\
0\\
\end{pmatrix}
+\begin{pmatrix}
i \alpha \hat{p}(r) \\
\frac{\partial \hat{p}}{\partial r} \\ 
\end{pmatrix}
= U \begin{pmatrix}
 \hat{u}_0 \\
 0 
\end{pmatrix}
\end{equation}

\end{document}