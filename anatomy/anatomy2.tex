\documentclass[12pt, letter]{report}

\usepackage{amsmath}    % need for subequations
\usepackage[pdftex]{graphicx}   % need for figures
\usepackage{verbatim}   % useful for program listings
\usepackage{color}      % use if color is used in text
\usepackage{caption}  
\usepackage{subcaption}  % use for side-by-side figures
\usepackage{hyperref}   % use for hypertext links, including those to external documents and URLs
\usepackage{setspace}
\doublespacing

\allowdisplaybreaks
\captionsetup{compatibility=false}

\graphicspath{{anatomy_figs/}}

\begin{document}
\section{Ethology of Rodent Ultrasonic Vocalizations}
To be a rodent is to be prey. A large number of vertebrate predators make use of rodents as a primary food source. Birds of prey are especially effective hunters, depleting local rodent populations in area before undertaking migrations to search for new hunting grounds. With these environmental pressure it is no wonder many rodents have evolved burrowing behaviors as a strategy for hiding from these dangers. However, specialized predators such as weasels have even evolved slender and elongated bodies to capture rodents from their burrows. Thus, even the relative safety of their nest areas can be ineffective protection. Some studies report that predation can have a staggering $95 \%$ impact on rodent populations \cite{Jedrzejewski1993}. With these considerations in mind, it is quite reasonable to expect that predation has been one of the driving factors in the evolution of rodent social behavior \cite{Brudzynski2010}. Perhaps one of the more powerful and complex evolutionary traits in vertebrates is that of vocal communication. As a results of their position in the food chain some rodent species have developed a peculiar form of vocal communication, one that lies entirely in the ultrasonic range. The advantage of this form of vocalization is the ability of an individual to communicate and provide warning to conspecifics with lowered risk of detection by predators. In this section I will discuss the origins of and behavior surrounding rodent ultrasonic vocalizations (USVs), more specifically those of rats. 

There is evidence that the phylogenetic origins of social vocalizations in vertebrates are over 400 million years old. Bass et al. have shown that the same vocal region in the caudal hindbrain and rostral spinal cord region is present in fish and all major lineages of vocal tetrapods \cite{Bass2008}. Rodent USV as a defensive adaptation is however a much more recent development. The suborder of myomorph rodent emerged about 40 million years ago, while the generas Mus and Rattus emerged as distinct groups between 16 and 23 million years ago. Species of both these generas use ultrasound for communication. Thus, it is likely rodent USV emerged somewhere between 20 and 40 million years ago. It is believed that the nocturnal lifestyle of the rodent predisposed them to evolve ultrasonic communication as a defensive strategy. Increased visual acuity is not beneficial to nocturnal animals. Thus, many rodents have evolved increased auditory acuity that extends into the ultrasonic range, and auditory sensitivity in the ultrasonic range is a prerequisite for communication in those frequencies.

It has been suggested by Blumberg and Alberts and then later by Hofer
and Shair that rodent USVs originally evolved as a way for infants to help
stimulate endogenous heat production. Blumberg and Alberts observed dur-
ing vocalization closure of the larynx results in an increase in intrathoracic
pressure, a phenomenon known as laryngeal braking. Under certain circum-
stances laryngeal braking can increase pulmonary oxygenation of the blood.
They hypothesized that this increased blood oxygen level could assist brown
adipose tissue with increasing the body temperature of a hypothermic infant
2
[4]. In their 1992 paper, Hofer and Shair observed that when infant rats are
brought into a comatose state of hypothermia, with a body of temperature of
20◦C, they will begin vocalizing in the ultrasonic range. When the comatose
rat pup is placed in contact with their mother there is initially no change in
the emission of USVs. However when their temperature warms to 25◦C they
become responsive to the mother’s presence and cease vocalizing [5]. This
suggests that infant USVs serve a dual purpose: to assist in raising the body
temperature of a hypothermic pup and to summon the mother to a pup in
distress. However, it does conclusively determine the original evolutionary
strategy surrounding infant USVs. To help elucidate this Hofer and Shair
compared rising body temperature in two groups of hypothermic rat pups,
a control group and a group devocalized through laryngeal denervation or
tracheostomy. They found that the body temperature of the control group
rose faster than the devocalized group, and 20% of the devocalized group did
not recover at all. Autopsy of these specimens found pulmonary edema [6].
Thus it seems likely that USVs in rodents originally evolved as a strategy
for increasing the body temperature in infant hypothermia and the maternal
instinct to interpret that as a distress call arose later [7].

\section{Anatomy of the Larynx}
The essential structure of the larynx is mostly conserved throughout mam-
mals. It consists of a cartilagenous framework, extrinsic laryngeal muscles,
3
which connect the framework to its surroundings in the body, and smaller
intrinsic laryngeal muscles, which lie entirely inside the larynx. The cartilage
framework consists of the cricoid, thyroid, and arytenoid cartilages (Fig. 1).
The cricoid cartilage is ring like in shape and sits at the top of the trachea.
The thyroid cartilage, also known as the Adam’s apple, attaches to the cricoid
cartilage posteriorly. It posses a few degrees of articulation. At the top of
cricoid cartilage ring sit the arytenoid cartilages. They are roughly pyrami-
dal in shape and have complex rotational articulations. Two sets of intrinsic
laryngeal muscles, the posterior and lateral cricoarytenoid muscles, attach
the arytenoid cartilages to the cricoid cartilage: and are responsible for their
abduction and adduction. Adduction of arytenoids occurs during swallowing,
to prevent debris from entering the trachea, and during phonation, to set the
vocal folds in a position in which they can oscillate. Adduction occurs when
the lateral cricoarytenoid muscles, attached to the front of the cricoid, con-
tract, causing the arytenoids to rotate inward. The interarytenoid muscles,
which as its name suggest lies in between the arytenoids assists with this
process. Abduction occurs during inspiration and occurs when the posterior
cricoarytenoid muscles contract rotating the arytenoids outward [8].
The thyroarytenoid muscles attach at the arytenoids anteriorly and run
to the thyroid cartilage posteriorly. These muscles are covered with a layer of
connective tissue. Collectively this muscles and connective tissue structure
are called the vocal folds. In common speech this structure is called the
vocal cords, which is somewhat of a misnomer, since they are not chord
4
(a)
 (b)
Figure 1: Ligaments of the larynx (a) antero-lateral view (b) posterior view
[9]
shape, nor do they generate sound in the manner of a vibrating guitar string.
As mentioned above, rotation of the arytenoids can abduct and adduct the
vocal folds by contraction of the cricoarytenoid and interarytenoid muscle.
Contraction of the thyroarytenoid muscles themselves can further adduct the
vocal folds, in addition to shortening them by rotating the arytenoids inward.
Contraction of the cricothyroid muscles can lengthen the vocal folds. A total
of five muscles contribute to the movement of and tension of the vocal folds.
Considering the small size of the structure this is a remarkable degree of
control, which is used for the complicated process of vocalization. Fig. 2
shows an illustration of the intrinsic laryngeal muscles [8].
Innervation of the vocal fold musculature is completely supplied by the
5
(a)
 (b)
Figure 2: (a) Top view of the laryngeal muscles (b) Side view of the laryngeal
muscles. Shown in both (a) and (b) are the thyroarytenoid muscles, the
interarytenoid muscles (arytanoideus), the posterior cricoarytenoid muscles,
and lateral cricoarytenoid muscles. Shown only in (b) is the cricothyroid
muscle [9].
6
vagus nerve. Efferent motor fibers of the nerve emerge from the nucleus am-
biguus in the medulla oblongata. Afferent sensory fibers of the nerve also
terminate there. At the root of the neck the vagus branches into left and
right components (?). Each component branches again into the superior
laryngeal nerve (SLN) and the recurrent laryngeal nerve (RLN). The RLN
contains only motor fibers, while the SLN contains both motor and sen-
sory fibers. The SLN runs to the larynx, where it divides into the external
and internal branches. The external branch provides motor control to the
cricothyroid muscle. The internal branch carries sensory information from
the mucosa lining, larynx, epiglottis, and part of the tongue. It enters the
larynx through the thyrohyoid membrane. The left and right RLNs branch
from the vagus nerve low in the neck. Upon reaching the larynx the RLNs
branch again, innervating the posterior cricothyroid muscles, lateral cricothy-
roid muscles, interarytenoid muscle, and the thyroarytenoid muscle [10]. Fig.
?? illustrates the path taking by the branchings of the vagus nerve. The or-
der of branching is stereotyped, but the precise branching pattern can vary
between individuals. Each laryngeal muscle can be individually controlled
by the human motor cortex. Such a high degree of control is indicative of
the large evolutionary advantage imparted by vocal communication [8].
Phonation is the physical process underlying vocalization. It is the pro-
cess by which the vocal folds produce sound through vibration. The basics
physics and physiology underlying phonation are as follows. Stimulation of
the intrinsic laryngeal muscles from the RLN causes the vocal folds to adduct
7
Figure 3: The vagus nerve [9]
into a closed position with a converging profile. With the vocal folds in a
closed position, air flow from the lungs results in an increase in subglottal
pressure. Because the of the converging profile, this increase in subglottal
pressure will force the vocal folds apart. The inferior portion of the vocal folds
open first. A wave, compressing the tissue laterally, travels upward to the
superior portion of the vocal fold. This is called a mucosal wave. The forcing
apart of the vocal folds causes an increase in airflow, which corresponds to a
drop in the pressure holding them open. Abduction will continue until this
pressure can longer balance the elastic restoring force of the vocal fold tissue.
At this point the vocal folds will begin to close. Their most inferior points
will collide before returning to their closed position with a converging profile,
when the cycle will start again [11, 8].
Detailed studies have been preformed indicating the sound properties of
vocalizations can be controlled through activity of the laryngeal muscles.
In most models the cricothyroid and thyroarytenoid are the most import
muscles for control of vocal acoustics. Studies have shown activation of the
cricothyroid muscle increases the fundamental frequency f0 of vibration by
increasing the internal tension of the vocal folds. The role of the thyroary-
tenoid muscle in the modulation of f0 is less clear. Studies in canines show f0
increases with thyroarytenoid activation when f0 is in the modal register but
decreases when f0 is the falsetto register [8, 12, 11]. The body-cover theory
proposed by Hirano and advanced by Titze attempts to explain modulations
of fundamental frequency as a function of cricothyroid and thyroarytenoid
muscle activation. The body-cover theory states that two distinct layers of
tissue contribute to vocal fold tension differently. The outer layer, the cover,
is composed of epithelial tissue as well as the superficial and intermediate
layers of the lamina propria. The cover has not contractile properties, and
thus it’s tension is completely controlled by the vocal fold length. The inner
group, the body, is composed of the deep layers of the lamina propria and
the thyroarytenoid muscle. It’s tension is not only determined by length but
also by the active contractile properties of the thyroarytenoid muscle. Thus
activation of the thyroarytenoid muscle can shorten the vocal folds but still
cause an increase in f0 due to an increase in internal stiffness of the muscle
fibers [13].
A simple rod model of the thyroid and cricoid cartilages with forces acting
on them due to laryngeal muscle activation leads to the following expression
9
for fundamental frequency

  12
Aa σam
f0 = f0p
 1+
 ata
A σp
σp = 4L20 (1 + )2f0p
 2
 ρ
f0p = 38 + 175( + 0.1) + 2450( + 0.1)2 + 2ps
 = 0.2 (1.5act − ata ) .
(1)
Here ata and act are the activation percentages of the thyroarytenoid and
cricothyroid muscles, f0p is the fundamental frequency for passive tissue (i.e
for ata = 0), σp is the mean passive stress over the entire vibrating cross
section A, σam is the maximum active stress in the muscular cross section
Aa , L0 is the passive length of the vocal folds,  is the vocal fold strain, ρ
is the vocal fold density, and ps is the subglottal pressure. The numerical
constants were obtained through experimental data fitting. A plot of f0 as
a function of thyroarytenoid and cricothyroid muscle activation can be seen
in Fig. ??. From the figure it can be seen that complex modulations of
fundamental frequency can be obtained through activation of these muscles.
Additionally it can be seen that while activation of the cricothyroid muscle
usually corresponds to an increase in f0 , activation of the thyroarytenoid
muscle can lead to more complicated behavior [13].
10
Figure 4: Fundamental frequency f0 plotted against thyroarytenoid muscle
activation ata for different values of cricothyroid muscle activation act ( A A a =
0.5, sigmap = 100 kPa, ps = 8 cm H20, L0 = 1.5 cm ). From this figure it
can be seen complex modulations of f0 can be performed through activation
of these muscles.
